\chapter{分支}

\section{逻辑运算符}

\subsection{关系运算符}

编程中经常需要使用关系运算符来比较两个数据的大小,比较的结果是一个布尔值(boolean),即True(非0)或False(0)。\\

在编程中需要注意,一个等号=表示赋值运算,而两个等号==表示比较运算。\\

\begin{table}[H]
	\centering
	\setlength{\tabcolsep}{5mm}{
		\begin{tabular}{|c|c|}
			\hline
			\textbf{数学符号} & \textbf{关系运算符} \\
			\hline
			$ < $             & <                   \\
			\hline
			$ > $             & >                   \\
			\hline
			$ \le $           & <=                  \\
			\hline
			$ \ge $           & >=                  \\
			\hline
			$ = $             & ==                  \\
			\hline
			$ \ne $           & !=                  \\
			\hline
		\end{tabular}
	}
	\caption{关系运算符}
\end{table}

\vspace{0.5cm}

\subsection{逻辑运算符}

逻辑运算符用于连接多个关系表达式,其结果也是一个布尔值。\\

\begin{enumerate}
	\item 逻辑与\&\&:当多个条件全部为True,结果为True。
	      \begin{table}[H]
		      \centering
		      \setlength{\tabcolsep}{5mm}{
			      \begin{tabular}{|c|c|c|}
				      \hline
				      \textbf{条件1} & \textbf{条件2} & \textbf{条件1 \&\& 条件2} \\
				      \hline
				      T              & T              & T                         \\
				      \hline
				      T              & F              & F                         \\
				      \hline
				      F              & T              & F                         \\
				      \hline
				      F              & F              & F                         \\
				      \hline
			      \end{tabular}
		      }
	      \end{table}

	\item 逻辑或||:多个条件至少有一个为True时,结果为True。
	      \begin{table}[H]
		      \centering
		      \setlength{\tabcolsep}{5mm}{
			      \begin{tabular}{|c|c|c|}
				      \hline
				      \textbf{条件1} & \textbf{条件2} & \textbf{条件1 || 条件2} \\
				      \hline
				      T              & T              & T                       \\
				      \hline
				      T              & F              & T                       \\
				      \hline
				      F              & T              & T                       \\
				      \hline
				      F              & F              & F                       \\
				      \hline
			      \end{tabular}
		      }
	      \end{table}

	\item 逻辑非!:条件为True时,结果为False;条件为False时,结果为True。
	      \begin{table}[H]
		      \centering
		      \setlength{\tabcolsep}{5mm}{
			      \begin{tabular}{|c|c|}
				      \hline
				      \textbf{条件} & \textbf{!条件} \\
				      \hline
				      T             & F              \\
				      \hline
				      F             & T              \\
				      \hline
			      \end{tabular}
		      }
	      \end{table}
\end{enumerate}

\newpage

\section{if}

\subsection{if}

if语句用于判断一个条件是否成立,如果成立则执行if语句块中的语句,否则不执行。\\

\vspace{-1cm}

\begin{lstlisting}[language=C]
#include <stdio.h>

int main()
{
	int age;
	printf("Enter your age: ");
	scanf("%d", &age);

	if(age > 0 && age < 18)
	{
		printf("minor\n");
	}

	return 0;
}
\end{lstlisting}

\vspace{0.5cm}

\subsection{if-else}

if-else的结构与if类似,只是在if语句块中的条件不成立时,执行else语句块中的语句。

\begin{lstlisting}[language=C]
#include <stdio.h>

int main()
{
	int age;
	printf("Enter your age: ");
	scanf("%d", &age);

	if(age > 0 && age < 18)
	{
		printf("minor\n");
	}
	else
	{
		printf("adult\n");
	}

	return 0;
}
\end{lstlisting}

\vspace{0.5cm}

\subsection{if-else if-else}

当需要对更多的条件进行判断时,可以使用if-else if-else语句。

\begin{lstlisting}[language=C]
#include <stdio.h>

int main()
{
	int age;
	printf("Enter your age: ");
	scanf("%d", &age);

	if(age > 0 && age < 18)
	{
		printf("minor\n");
	}
	else if(age >= 18 && age <= 30)
	{
		printf("yound adult\n");
	}
	else if(age >= 31 && age <= 60)
	{
		printf("middle-aged\n");
	}
	else if(age >= 61 && age <= 120)
	{
		printf("senior\n");
	}
	else
	{
		printf("invalid age\n");
	}
	
	return 0;
}
\end{lstlisting}

\newpage

\section{switch}

\subsection{switch}

switch结构用于根据一个整数值,选择对应的case执行。需要注意的是,当对应的case中的代码被执行完后,并不会像if语句一样跳出switch结构,而是会继续向后执行,直到遇到break。\\

\mybox{计算器}

\begin{lstlisting}[language=C]
#include <stdio.h>

int main() {
	int num1, num2;
	char operator;

	printf("Enter an expression: ");
	scanf("%d %c %d", &num1, &operator, & num2);

	switch (operator)
	{
	case '+':
		printf("%d + %d = %d\n", num1, num2, num1 + num2);
		break;
	case '-':
		printf("%d - %d = %d\n", num1, num2, num1 - num2);
		break;
	case '*':
		printf("%d * %d = %d\n", num1, num2, num1 * num2);
		break;
	case '/':
		printf("%d / %d = %d\n", num1, num2, num1 / num2);
		break;
	default:
		printf("Error! Operator is not supported\n");
		break;
	}

	return 0;
}
\end{lstlisting}

\begin{tcolorbox}
	\mybox{运行结果}
	\begin{verbatim}
Enter an expression: 5 * 8
5 * 8 = 40
	\end{verbatim}
\end{tcolorbox}

\newpage