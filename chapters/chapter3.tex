\chapter{判断}

\section{逻辑运算符}

\subsection{关系运算符}

\begin{table}[H]
	\centering
	\setlength{\tabcolsep}{5mm}{
		\begin{tabular}{|c|c|}
			\hline
			\textbf{数学符号} & \textbf{关系运算符} \\
			\hline
			$ < $             & <                   \\
			\hline
			$ > $             & >                   \\
			\hline
			$ \le $           & <=                  \\
			\hline
			$ \ge $           & >=                  \\
			\hline
			$ \ne $           & !=                  \\
			\hline
			$ = $             & ==                  \\
			\hline
		\end{tabular}
	}
	\caption{关系运算符}
\end{table}

\vspace{0.5cm}

\subsection{逻辑运算符}

Java中逻辑运算符有三种:

\begin{enumerate}
	\item 逻辑与\&\&(logical AND):当多个条件同时为真,结果为真。
	      \begin{table}[H]
		      \centering
		      \setlength{\tabcolsep}{5mm}{
			      \begin{tabular}{|c|c|c|}
				      \hline
				      \textbf{条件1} & \textbf{条件2} & \textbf{条件1 \&\& 条件2} \\
				      \hline
				      T              & T              & T                         \\
				      \hline
				      T              & F              & F                         \\
				      \hline
				      F              & T              & F                         \\
				      \hline
				      F              & F              & F                         \\
				      \hline
			      \end{tabular}
		      }
		      \caption{逻辑与}
	      \end{table}

	\item 逻辑或||(logical OR):多个条件有一个为真时,结果为真。
	      \begin{table}[H]
		      \centering
		      \setlength{\tabcolsep}{5mm}{
			      \begin{tabular}{|c|c|c|}
				      \hline
				      \textbf{条件1} & \textbf{条件2} & \textbf{条件1 || 条件2} \\
				      \hline
				      T              & T              & T                       \\
				      \hline
				      T              & F              & T                       \\
				      \hline
				      F              & T              & T                       \\
				      \hline
				      F              & F              & F                       \\
				      \hline
			      \end{tabular}
		      }
		      \caption{逻辑或}
	      \end{table}

	\item 逻辑非!(logical NOT):条件为真时,结果为假;条件为假时,结果为真。
	      \begin{table}[H]
		      \centering
		      \setlength{\tabcolsep}{5mm}{
			      \begin{tabular}{|c|c|}
				      \hline
				      \textbf{条件} & \textbf{!条件} \\
				      \hline
				      T             & F              \\
				      \hline
				      F             & T              \\
				      \hline
			      \end{tabular}
		      }
		      \caption{逻辑非}
	      \end{table}
\end{enumerate}

\newpage

\section{if}

\subsection{if}

当if语句的条件为真时,进入花括号执行内部的代码;若条件为假,则跳过花括号执行后面的代码。\\

if语句主要有以下几种形式:

\subsubsection{单分支}

\vspace{-1cm}

\begin{lstlisting}[language=Java]
public class IfStmt {
	public static void main(String[] args) {
		int age = 15;
		if(age > 0 && age < 18) {
			System.out.println("未成年");
		}
	}
}
\end{lstlisting}

\subsubsection{双分支}

\vspace{-1cm}

\begin{lstlisting}[language=Java]
public class IfElse {
	public static void main(String[] args) {
		int age = 30;
		if(age > 0 && age < 18) {
			System.out.println("未成年人");
		} else {
			System.out.println("成年人");
		}
	}
}
\end{lstlisting}

\subsubsection{多分支}

\vspace{-1cm}

\begin{lstlisting}[language=Java]
public class IfElseifElse {
	public static void main(String[] args) {
		int score = 76;
			
		if(score >= 90 && score <= 100) {
			System.out.println("优秀");
		} else if(score >= 60) {
			System.out.println("合格");
		} else {
			System.out.println("不合格");
		}
	}
}
\end{lstlisting}

\vspace{0.5cm}

\subsection{嵌套结构}

if语句也可以嵌套使用:

\vspace{-0.5cm}

\begin{lstlisting}[language=Java]
if(条件1) {
	if(条件2) {
		// code
	}
}
\end{lstlisting}

\vspace{0.5cm}

\mybox{判断整数奇偶}

\begin{lstlisting}[language=Java]
import java.util.Scanner;

public class OddEven {
	public static void main(String[] args) {
		Scanner scanner = new Scanner(System.in);
		int num;
		
		System.out.print("输入一个正整数:");
		num = scanner.nextInt();
		
		if(num > 0) {
			if(num % 2 == 0) {
				System.out.println(num + "是偶数");
			} else {
				System.out.println(num + "是奇数");
			}
		}
		
		scanner.close();
	}
}
\end{lstlisting}

\begin{tcolorbox}
	\mybox{运行结果}
	\begin{verbatim}
输入一个正整数:66
66是偶数
	\end{verbatim}
\end{tcolorbox}

\newpage

\section{switch}

\subsection{switch}

switch-case结构可以对整数值的表达式进行判断。

\vspace{-0.5cm}

\begin{lstlisting}[language=Java]
switch(表达式) {
    case label:
        //code
        break;
    // ...
    default:
        //code
        break;
}
\end{lstlisting}

根据表达式的值,跳转到对应的case处进行执行。需要注意的是,当对应的case中的代码被执行完后,并不会跳出switch,而是会继续执行后面的代码,所以需要使用break跳出switch结构。当所有case都不满足表达式的值时,会执行default语句中的代码,相当于if-else结构中的else。\\

\mybox{根据月份输出对应的英语简写}

\begin{lstlisting}[language=Java]
import java.util.Scanner;

public class Month {
	public static void main(String[] args) {
		Scanner scanner = new Scanner(System.in);
		int month;
		System.out.print("输入月份:");
		month = scanner.nextInt();
		
		switch(month) {
			case 1:
				System.out.println("Jan.");
				break;
			case 2:
				System.out.println("Feb.");
				break;
			case 3:
				System.out.println("Mar.");
				break;
			case 4:
				System.out.println("Apr.");
				break;
			case 5:
				System.out.println("May");
				break;
			case 6:
				System.out.println("Jun.");
				break;
			case 7:
				System.out.println("Jul.");
				break;
			case 8:
				System.out.println("Aug.");
				break;
			case 9:
				System.out.println("Sep.");
				break;
			case 10:
				System.out.println("Oct.");
				break;
			case 11:
				System.out.println("Nov.");
				break;
			case 12:
				System.out.println("Dec.");
				break;
			default:
				System.out.println("输入有误");
				break;
		}
		scanner.close();
	}
}
\end{lstlisting}

\begin{tcolorbox}
	\mybox{运行结果}
	\begin{verbatim}
输入月份:5
May
	\end{verbatim}
\end{tcolorbox}

\newpage