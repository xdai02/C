\chapter{C简介}

\section{编程简介}

\subsection{编程简介}

程序(program)是为了让计算机执行某些操作或者解决问题而编写的一系列有序指令的集合。由于计算机只能够识别二进制数字0和1,因此需要使用特殊的编程语言来描述如何解决问题过程和方法。\\

算法(algorithm)是可完成特定任务的一系列步骤,算法的计算过程定义明确,通过一些值作为输入并产生一些值作为输出。\\

流程图(flow chart)是算法的一种图形化表示方式,使用一组预定义的符号来说明如何执行特定任务。

\begin{itemize}
	\item 圆角矩形:开始和结束
	\item 矩形:数据处理
	\item 平行四边形:输入/输出
	\item 菱形:分支判断条件
	\item 流程线:步骤
\end{itemize}

\begin{figure}
	\centering
	\begin{tikzpicture}[node distance=2cm]
		\node (start) [startend] {Start};
		\node (init)   [io, below of=start] {$ i = 0 $, $ sum = 0 $};
		\node (decision)  [decision, below of=init] {$ i \le 100 $?};
		\node (accumulation) [process, below of=decision] {$ sum = sum + i $};
		\node (update) [process, below of=accumulation] {$ i = i + 1 $};
		\node (output) [io, right of=decision, xshift=2.5cm] {print $ sum $};
		\node (end) [startend, below of=update] {End};

		\draw [arrow] (start) -- (init);
		\draw [arrow] (init) -- (decision);
		\draw [arrow] (decision) -- node[anchor=east] {yes } (accumulation);
		\draw [arrow] (accumulation) -- (update);
		\draw [arrow] (update) -- (-3,-8) -- (-3,-4) -- (decision);
		\draw [arrow] (decision) -- node[anchor=south] {no} (output);
		\draw [arrow] (output) |- (end);
	\end{tikzpicture}
	\caption{计算$ \sum_{i=1}^{100} i $的流程图}
\end{figure}

\vspace{0.5cm}

\subsection{编程语言(Programming Language)}

编程语言主要分为面向机器、面向过程和面向对象三类。C是面向过程的语言,常用于操作系统、嵌入式系统、驱动程序、图形引擎、图像处理、声音效果等。\\

C是一个结构化的编程语言,因此它层次清晰便于按模块化方式组织程序,易于调试和维护。然而结构化的缺点也很明显,比如程序的可重用性差。

\begin{figure}[H]
	\centering
	\includegraphics[scale=0.9]{img/C1/1-1/1.png}
	\caption{常见编程语言}
\end{figure}

\newpage

\section{Hello World!}

\subsection{Hello World!}

\mybox{Hello World!}

\begin{lstlisting}[language=C]
#include <stdio.h>

int main()
{
	printf("Hello World!\n");
	return 0;
}
\end{lstlisting}

\begin{tcolorbox}
	\mybox{运行结果}
	\begin{verbatim}
Hello World!
	\end{verbatim}
\end{tcolorbox}

第一行语句为预处理指令,预处理指令以【\#】开头,\#include表示程序需要使用头文件stdio.h中的函数,stdio.h文件中包含了有关输入输出语句的函数。\\

main()是C程序的入口,在程序的主体部分,printf()的功能是在屏幕上输出"Hello World"这个字符串,$ \backslash $n表示换行符。\\

C中【;】表示语句结束,注意不要使用中文的分号。一条语句可以跨越多行,并且用分号通知编译器该语句结束。\\

编译器(compiler)将程序编译成计算机能够识别的二进制文件,最终能够被计算机执行。

\newpage

\section{Error or Warning?}

\subsection{Error / Warning}

在编写程序的过程中,错误是不可避免的,错误主要能够分为以下三种类别:

\begin{enumerate}
	\item 语法错误(syntax error):程序的语法不合符编程语言的要求,编译器会反馈报错信息。

	\item 逻辑错误(logical error):人类在编程过程中的逻辑错误,无法被编译器所检测。

	\item 运行时错误(runtime error)例如除以0、数组越界、指针越界、使用已经释放的空间、栈溢出等情况,可以被编译器发现。
\end{enumerate}

\newpage

\section{注释}

\subsection{注释(Comment)}

在编程中加入注释可以增加程序的可读性和可维护性,编译器不会对注释的部分进行编译。\\

C中注释分为两类:

\begin{enumerate}
	\item 单行注释:将一行内【//】之后的内容视为注释。
	\item 多行注释:以【/*】开始,【*/】结束,中间的内容视为注释。
\end{enumerate}

\mybox{注释}

\begin{lstlisting}[language=C]
/*
    这个程序在屏幕上是输出Hello World
*/
#include <stdio.h>              // 头文件

int main()
{
    printf("Hello World!\n");   // 输出
    return 0;
}
\end{lstlisting}

\begin{tcolorbox}
	\mybox{运行结果}
	\begin{verbatim}
Hello World!
	\end{verbatim}
\end{tcolorbox}

\newpage

\section{不同语言的Hello World}

\subsection{编程语言对比}

\mybox{C++}

\begin{lstlisting}[language=C++]
#include <iostream>
using namespace std;

int main() {
	cout << "Hello World" << endl;
	return 0;
}
\end{lstlisting}

\vspace{0.5cm}

\mybox{Java}

\begin{lstlisting}[language=Java]
public class HelloWorld {
    public static void main(String[] args) {
        System.out.println("Hello World");
    }
}
\end{lstlisting}

\vspace{0.5cm}

\mybox{Python}

\begin{lstlisting}[language=Python]
print("Hello World")
\end{lstlisting}

\newpage