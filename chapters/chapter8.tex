\chapter{文件}

\section{文件}

\subsection{文件打开}

C具有操作文件的能力,比如对文件数据的添加、删除、修改等。为了统一对各种硬件的操作,不同的硬件设备也都被看作是文件进行管理。计算机中键盘是标准输入设备stdin,显示器是标准输出设备stdout。\\

C通过声明一个文件FILE类型的指针,可以对指针所指向的文件进行操作。\\

fopen()用于打开文件,函数原型为:

\vspace{-0.5cm}

\begin{lstlisting}[language=C]
FILE *fopen(const char *fname, const char *mode);
\end{lstlisting}

\begin{table}[H]
	\centering
	\setlength{\tabcolsep}{5mm}{
		\begin{tabular}{|c|l|}
			\hline
			\textbf{打开方式} & \textbf{描述}                                         \\
			\hline
			r                 & 只读,文件必须存在,否则打开失败                      \\
			\hline
			w                 & 只写,创建一个新文件                                  \\
			\hline
			a                 & 追加,如果文件不存在则创建;存在则将数据追加到末尾    \\
			\hline
			r+                & 读+写,文件必须存在,否则打开失败                     \\
			\hline
			w+                & 写+读,创建一个新文件                                 \\
			\hline
			a+                & 追加+读,如果文件不存在则创建;存在则将数据追加到末尾 \\
			\hline
			rb                & 以只读打开二进制文件                                  \\
			\hline
			wb                & 以只写打开二进制文件                                  \\
			\hline
			ab                & 以追加打开二进制文件                                  \\
			\hline
			rb+               & 以读+写打开二进制文件                                 \\
			\hline
			wb+               & 以写+读打开二进制文件                                 \\
			\hline
			ab+               & 以追加+读打开二进制文件                               \\
			\hline
		\end{tabular}
	}
	\caption{文件打开方式}
\end{table}

如果文件打开失败,fopen()则会返回空指针NULL。\\

\subsection{文件关闭}

在对文件操作结束后,需要使用fclose()将文件关闭。\\

fclose()负责清空缓冲区,并释放文件指针。需要特别注意的是,在对文件执行写操作以后,并不会马上写入文件,而只是写入到了这个文件的输出缓冲区中。只有当输出缓冲区满了,或者执行了fflush(),或者执行了fclose()以后,或者程序结束,才会把输出缓冲区中的内容写入文件。\\

\mybox{文件}

\begin{lstlisting}[title=data.txt]
This is a test.
\end{lstlisting}

\begin{lstlisting}[language=C, title=file\_open.c]
#include <stdio.h>
#include <stdlib.h>

int main() {
    FILE *fp = fopen("data.txt", "r");
    if(!fp) {
        fprintf(stderr, "File Open Failed\n");
        exit(1);
    }
    fclose(fp);
    return 0;
}
\end{lstlisting}

\newpage

\section{文件读写}

\subsection{fgetc()读字符}

fgetc()的功能是从文件读取一个字符。成功时,返回读到的字符(int类型);失败或读到文件尾,返回EOF(-1)。\\

fgetc()函数原型:

\vspace{-0.5cm}

\begin{lstlisting}[language=C]
int fgetc(FILE *stream);
\end{lstlisting}

\vspace{0.5cm}

\mybox{读取并输出指定文件内容}

\begin{lstlisting}[title=data.txt]
极夜酱	17
小灰	22
小白	19
\end{lstlisting}

\begin{lstlisting}[language=C, title=fgetc.c]
#include <stdio.h>
#include <stdlib.h>

int main() {
    FILE *fp = fopen("data.txt", "r");
    if(!fp) {
        fprintf(stderr, "File Open Failed\n");
        exit(1);
    }

    char c;
    while((c = fgetc(fp)) != EOF) {
        printf("%c", c);
    }
    
    fclose(fp);
    return 0;
}
\end{lstlisting}

\vspace{0.5cm}

\mybox{统计程序源代码的字符数和行数}

\begin{lstlisting}[language=C, title=count\_chars\_and\_lines.c]
#include <stdio.h>
#include <stdlib.h>

int main() {
    FILE *fp = fopen("count_chars_and_lines.c", "r");
    if(!fp) {
        fprintf(stderr, "File Open Failed\n");
        exit(1);
    }

    char c;
    int charNum = 0;        // 字符数量
    int lineNum = 0;        // 行数

    while((c = fgetc(fp)) != EOF) {
        if(c == '\n') {
            lineNum++;
        } else {
            charNum++;
        }
    }

    printf("字符数:%d\n", charNum);
    printf("行  数:%d\n", lineNum);

    fclose(fp);
    return 0;
}
\end{lstlisting}

\begin{tcolorbox}
	\mybox{运行结果}
	\begin{verbatim}
字符数:513
行  数:27
	\end{verbatim}
\end{tcolorbox}

\vspace{0.5cm}

\subsection{fputc()写字符}

fputc()的功能是将一个字符写入文件中。\\

fputc()函数原型:

\vspace{-0.5cm}

\begin{lstlisting}[language=C]
int fputc(int ch, FILE *stream);
\end{lstlisting}

\vspace{0.5cm}

\mybox{将程序源代码输出到指定文件}

\begin{lstlisting}[language=C, title=fputc.c]
#include <stdio.h>
#include <stdlib.h>

int main() {
    FILE *fp1 = fopen("fputc.c", "r");
    FILE *fp2 = fopen("data.txt", "w");
    if(!fp1) {
        fprintf(stderr, "File Open Failed\n");
        exit(1);
    }

    char c;
    while((c = fgetc(fp1)) != EOF) {
        fputc(c, fp2);
    }

    fclose(fp1);
    fclose(fp2);
    return 0;
}
\end{lstlisting}

\vspace{0.5cm}

\subsection{fgets()读字符串}

fgets()的功能是从文件读取一个字符串。读取成功时,返回指向字符串的指针;读取失败时,返回NULL。\\

fgets()函数原型:

\vspace{-0.5cm}

\begin{lstlisting}[language=C]
char *fgets(char *str, int num, FILE *stream);
\end{lstlisting}

\begin{itemize}
	\item str:用于保存字符串的变量。
	\item num:最多读取字符数量,由于字符串结尾需要保留$ \backslash $0结束符,因此真正只能最多读取num - 1个字符。
\end{itemize}

\vspace{0.5cm}

\mybox{读取并输出程序源代码内容}

\begin{lstlisting}[language=C, title=fgets.c]
#include <stdio.h>
#include <stdlib.h>

int main() {
    FILE *fp = fopen("data.txt", "r");
    if(!fp) {
        fprintf(stderr, "File Open Failed\n");
        exit(1);
    }

    char c;
    while((c = fgetc(fp)) != EOF) {
        printf("%c", c);
    }
    
    fclose(fp);
    return 0;
}
\end{lstlisting}

\vspace{0.5cm}

\subsection{fputs()写字符串}

fputs()的功能是将一个字符串写入文件中。\\

fputs()函数原型:

\vspace{-0.5cm}

\begin{lstlisting}[language=C]
char *fputs(const char *str, FILE *stream);
\end{lstlisting}

\vspace{0.5cm}

\mybox{将程序源代码输出到指定文件}

\begin{lstlisting}[language=C, title=fputs.c]
#include <stdio.h>
#include <stdlib.h>

int main() {
    FILE *fp1 = fopen("fputs.c", "r");
    FILE *fp2 = fopen("data.txt", "w");
    if(!fp1) {
        fprintf(stderr, "File Open Failed\n");
        exit(1);
    }

    char line[128];
    while(fgets(line, sizeof(line), fp1)) {
        fputs(line, fp2);
    }

    fclose(fp1);
    fclose(fp2);
    return 0;
}
\end{lstlisting}

\vspace{0.5cm}

\subsection{fprintf()格式化输出}

fprintf()使用方法与printf()类似,只是多增加了一个参数,用于指定输出流。\\

fprintf()函数原型:

\vspace{-0.5cm}

\begin{lstlisting}[language=C]
int fprintf(FILE *stream, const char *format, ...);
\end{lstlisting}

\vspace{0.5cm}

\mybox{将数据格式化输出到指定文件}

\begin{lstlisting}[language=C, title=fprintf.c]
#include <stdio.h>
#include <stdlib.h>

int main() {
    FILE *fp = fopen("data.txt", "w");
    char name[3][12] = {"极夜酱", "小灰", "小白"};
    int age[3] = {17, 22, 19};

    for(int i = 0; i < 3; i++) {
        fprintf(fp, "%s\t%d\n", name[i], age[i]);
    }
    
    fclose(fp);
    return 0;
}
\end{lstlisting}

\begin{tcolorbox}
	\mybox{运行结果}
	\textbf{data.txt}
	\begin{verbatim}
极夜酱	17
小灰	22
小白	19
	\end{verbatim}
\end{tcolorbox}

\vspace{0.5cm}

\subsection{fscanf()格式化输入}

fscanf()的功能是按照指定格式从文件读取数据。读取成功时返回实际读取的数据个数,失败时返回EOF。\\

fscanf()函数原型:

\vspace{-0.5cm}

\begin{lstlisting}[language=C]
int fscanf(FILE *stream, const char *format, ...);
\end{lstlisting}

\vspace{0.5cm}

\mybox{从指定文件读取指定格式数据}

\begin{lstlisting}[title=data.txt]
极夜酱	17
小灰	22
小白	19
\end{lstlisting}

\begin{lstlisting}[language=C, title=fscanf.c]
#include <stdio.h>
#include <stdlib.h>

int main() {
    FILE *fp = fopen("data.txt", "r");
    if(!fp) {
        fprintf(stderr, "File Open Failed\n");
        exit(1);
    }
    
    char name[12];
    int age;
    
    while(fscanf(fp, "%s\t%d", name, &age) != EOF) {
        printf("%s\t%d\n", name, age);
    }

    fclose(fp);
    return 0;
}
\end{lstlisting}

\begin{tcolorbox}
	\mybox{运行结果}
	\begin{verbatim}
极夜酱	17
小灰	22
小白	19
	\end{verbatim}
\end{tcolorbox}

\vspace{0.5cm}

\subsection{feof()检查文件结束}

feof()的功能是检查文件是否已经达到文件末尾位置,如果是就返回非零值(真)。\\

feof()函数原型:

\vspace{-0.5cm}

\begin{lstlisting}[language=C]
int feof(FILE *stream);
\end{lstlisting}

\vspace{0.5cm}

\mybox{从通讯录文件中查找指定人名}

\begin{lstlisting}[title=data.txt]
极夜酱	17
小灰	22
小白	19
\end{lstlisting}

\begin{lstlisting}[language=C, title=feof.c]
#include <stdio.h>
#include <stdlib.h>
#include <string.h>
#include <stdbool.h>

int main() {
    FILE *fp = fopen("data.txt", "r");
    if(!fp) {
        fprintf(stderr, "File Open Failed\n");
        exit(1);
    }

    char key[32];       // 需查找数据
    char name[32];
    int age;
    bool found = false; // 是否找到

    printf("查找姓名:");
    gets(key);

    while(!feof(fp)) {
        fscanf(fp, "%s\t%d", name, &age);
        if(strcmp(name, key) == 0) {
            printf("%s\t%d\n", name, age);
            found = true;
            break;
        }
    }

    if(!found) {
        printf("未找到【%s】的信息\n", key);
    }

    fclose(fp);
    return 0;
}
\end{lstlisting}

\begin{tcolorbox}
	\mybox{运行结果}
	\begin{verbatim}
查找姓名:小灰
小灰	22
	\end{verbatim}
\end{tcolorbox}

\newpage