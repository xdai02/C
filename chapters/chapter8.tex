\chapter{文件}

\section{文件}

\subsection{fopen()}

文件是存储数据的一种常用方式,程序可以从文件中读取和写入数据,从而实现对数据的持久化存储。\\

在对文件进行操作之前,首先需要使用fopen()函数打开文件,fopen()的函数原型为:

\vspace{-0.5cm}

\begin{lstlisting}[language=C]
FILE *fopen(const char *filename, const char *mode);
\end{lstlisting}

fopen()接受两个参数,第一个参数是要打开的文件名,第二个参数为打开方式。fopen()会返回一个FILE类型的指针,通过该指针可以对文件进行操作;如果文件打开失败,则返回NULL。\\

\begin{table}[H]
    \centering
    \setlength{\tabcolsep}{5mm}{
        \begin{tabular}{|c|l|}
            \hline
            \textbf{打开方式} & \textbf{功能}                                      \\
            \hline
            r                 & 只读,文件必须存在,否则打开失败                   \\
            \hline
            w                 & 只写,创建一个新文件                               \\
            \hline
            a                 & 追加,如果文件不存在则创建;存在则将数据追加到末尾 \\
            \hline
            r+                & 以r模式打开文件,附带写的功能                      \\
            \hline
            w+                & 以w模式打开文件,附带读的功能                      \\
            \hline
            a+                & 以a模式打开文件,附带读的功能                      \\
            \hline
        \end{tabular}
    }
    \caption{文件打开方式}
\end{table}

\vspace{0.5cm}

\subsection{fclose()}

在对文件操作结束后,需要使用fclose()函数将文件关闭。fclose()函数原型:

\vspace{-0.5cm}

\begin{lstlisting}[language=C]
int fclose(FILE *stream);
\end{lstlisting}

\vspace{0.5cm}

\mybox{文件}

\begin{lstlisting}[title=data.txt]
This is a test.
\end{lstlisting}

\begin{lstlisting}[language=C]
#include <stdio.h>
#include <stdlib.h>

int main() {
    FILE *fp = fopen("data.txt", "r");
    if(!fp) {
        exit(1);
    }
    fclose(fp);
    return 0;
}
\end{lstlisting}

\newpage

\section{文件I/O}

\subsection{fprintf()}

fprintf()函数用于将数据输出到文件中,使用方法与printf()类似。fprintf()的函数原型为:

\vspace{-0.5cm}

\begin{lstlisting}[language=C]
int fprintf(FILE *stream, const char *format, ...);
\end{lstlisting}

\vspace{0.5cm}

\mybox{成绩}

\begin{lstlisting}[language=C]
#include <stdio.h>

int main() {
    FILE *fp = fopen("data1.txt", "w");
    int n;
    printf("Enter number of students: ");
    scanf("%d", &n);

    char id[8];
    double score;
    for (int i = 0; i < n; i++) {
        printf("Enter student %d's ID: ", i + 1);
        scanf("%s", id);
        printf("Enter student %d's score: ", i + 1);
        scanf("%lf", &score);
        fprintf(fp, "ID=%s\tScore=%.2f\n", id, score);
    }

    fclose(fp);
    return 0;
}
\end{lstlisting}

\begin{tcolorbox}
    \mybox{运行结果}
    \begin{verbatim}
Enter number of students: 5
Enter student 1's ID: A001
Enter student 1's score: 92
Enter student 2's ID: A002
Enter student 2's score: 73
Enter student 3's ID: A003
Enter student 3's score: 89
Enter student 4's ID: A004
Enter student 4's score: 97
Enter student 5's ID: A005
Enter student 5's score: 65
	\end{verbatim}
\end{tcolorbox}

\begin{tcolorbox}
    \mybox{运行结果}
    \textbf{data1.txt}
    \begin{verbatim}
ID=A001	Score=92.00
ID=A002	Score=73.00
ID=A003	Score=89.00
ID=A004	Score=97.00
ID=A005	Score=65.00
	\end{verbatim}
\end{tcolorbox}

为了统一对各种硬件的操作,不同的硬件设备也都被看作是文件进行管理。计算机中标准输入(stdin)是键盘、标准输出(stdout)是显示器、标准错误(stderr)是显示器。\\

因此,当使用printf()函数时,其实是从将数据输出到显示器上。printf()函数是通过调用fprintf(stdout, ...)来实现的。\\

当需要输出一些错误信息时,可以通过fprintf(stderr, ...)将错误信息输出到标准错误stderr上,这样可以避免将错误信息混入到正常的输出信息中,方便查看和分析。\\

\subsection{fscanf()}

fscanf()函数用于从文件中读取数据,使用方法与scanf()类似。fscanf()的函数原型为:

\vspace{-0.5cm}

\begin{lstlisting}[language=C]
int fscanf(FILE *stream, const char *format, ...);
\end{lstlisting}

fscanf()函数读取成功时返回实际读取的数据个数,失败时返回文件末尾标志EOF(End of File)。\\

\mybox{平均分}

\begin{lstlisting}[language=C]
#include <stdio.h>
#include <stdlib.h>

int main() {
    FILE *fp = fopen("data1.txt", "r");
    if (!fp) {
        fprintf(stderr, "File open failed.\n");
        exit(1);
    }

    char id[8];
    double score;
    double sum = 0;
    int n = 0;

    while (fscanf(fp, "ID=%s\tScore=%lf\n", id, &score) != EOF) {
        sum += score;
        n++;
    }
    printf("Average = %.2f\n", sum / n);

    fclose(fp);
    return 0;
}
\end{lstlisting}

\begin{tcolorbox}
    \mybox{运行结果}
    \begin{verbatim}
Average = 83.20
	\end{verbatim}
\end{tcolorbox}

当使用scanf()函数时,其实是从键盘上读取数据的,scanf()函数是通过调用fscanf(stdin, ...)来实现的。\\

\subsection{fputc()}

fputc()函数用于将一个字符写入文件中,其函数原型为:

\vspace{-0.5cm}

\begin{lstlisting}[language=C]
int fputc(int ch, FILE *stream);
\end{lstlisting}

fputc()接受两个参数,第一个参数为要写入的字符(ASCII码),第二个参数为文件指针。\\

当向屏幕输出一个字符时,fputc(stdout)等价于putchar()。\\

\mybox{大写字母}

\begin{lstlisting}[language=C]
#include <stdio.h>

int main() {
    FILE *fp = fopen("data2.txt", "w");

    for (int i = 0; i < 26; i++) {
        fputc('A' + i, fp);
    }

    fclose(fp);
    return 0;
}
\end{lstlisting}

\begin{tcolorbox}
    \mybox{运行结果}
    \textbf{data2.txt}
    \begin{verbatim}
ABCDEFGHIJKLMNOPQRSTUVWXYZ
	\end{verbatim}
\end{tcolorbox}

\vspace{0.5cm}

\subsection{fgetc()}

fgetc()函数用于从文件中读取一个字符,读取成功返回字符的ASCII码,失败返回EOF。fgetc()的函数原型为:

\vspace{-0.5cm}

\begin{lstlisting}[language=C]
int fgetc(FILE *stream);
\end{lstlisting}

当从键盘读取一个字符时,fgetc(stdin)等价于getchar()。\\

在读取文件时,除了可以通过返回值EOF来判断是否读取到文件末尾外,还可以使用feof()函数,当读到文件末尾时返回非0值,否则返回0。feof()的函数原型为:

\vspace{-0.5cm}

\begin{lstlisting}[language=C]
int feof(FILE *stream);
\end{lstlisting}

\vspace{0.5cm}

\mybox{源代码统计}

\begin{lstlisting}[language=C]
#include <stdio.h>
#include <stdlib.h>

int main() {
    FILE *fp = fopen("statistics.c", "r");
    if (!fp) {
        fprintf(stderr, "File open failed.\n");
        exit(1);
    }

    int chars = 0;
    int lines = 0;

    while (!feof(fp)) {
        char c = fgetc(fp);
        if (c == '\n') {
            lines++;
        } else {
            chars++;
        }
    }

    printf("Characters: %d\n", chars);
    printf("Lines: %d\n", lines);

    fclose(fp);
    return 0;
}
\end{lstlisting}

\begin{tcolorbox}
    \mybox{运行结果}
    \begin{verbatim}
Characters: 454
Lines: 27
	\end{verbatim}
\end{tcolorbox}

\vspace{0.5cm}

\subsection{fputs()}

fputs()函数用于将一个字符串写入文件,其函数原型为:

\vspace{-0.5cm}

\begin{lstlisting}[language=C]
int fputs(const char *str, FILE *stream);
\end{lstlisting}

当向屏幕输出一个字符串时,fputs(stdout)等价于puts()。\\

\mybox{Computer Science Quotes}

\begin{lstlisting}[language=C]
#include <stdio.h>

int main() {
    const char *quotes[] = {
        "Talk is cheap. Show me the code.",
        "Code never lies, comments sometimes do.",
        "Stay Hungry Stay Foolish.",
    };
    int n = sizeof(quotes) / sizeof(quotes[0]);

    FILE *fp = fopen("data3.txt", "w");

    for (int i = 0; i < n; i++) {
        fputs(quotes[i], fp);
        fputc('\n', fp);
    }

    fclose(fp);
    return 0;
}
\end{lstlisting}

\begin{tcolorbox}
    \mybox{运行结果}
    \textbf{data3.txt}
    \begin{verbatim}
Talk is cheap. Show me the code.
Code never lies, comments sometimes do.
Stay Hungry Stay Foolish.
	\end{verbatim}
\end{tcolorbox}

\vspace{0.5cm}

\subsection{fgets()}

fgets()函数用于从文件读取一行数据,读取成功返回指向字符串的指针,失败则返回NULL。fgets()的函数原型为:

\vspace{-0.5cm}

\begin{lstlisting}[language=C]
char *fgets(char *str, int n, FILE *stream);
\end{lstlisting}

fgets()接受三个参数,第一个参数用于保存读取到的字符串,第二个参数用于指定读取的最大字符数(包括结尾的$ \backslash $0),第三个参数为文件指针。\\

\mybox{解析单词}

\begin{lstlisting}[language=C]
#include <stdio.h>
#include <stdlib.h>
#include <string.h>
#include <ctype.h>

int main() {
    FILE *fp = fopen("data3.txt", "r");
    if (!fp) {
        fprintf(stderr, "File open failed.\n");
    }

    char line[128];
    while (fgets(line, sizeof(line), fp) != NULL) {
        char *token = strtok(line, " \t\n");
        while (token != NULL) {
            // remove punctuations
            int i = strlen(token) - 1;
            while (!isalpha(token[i])) {
                i--;
            }
            token[i + 1] = '\0';

            printf("%s\n", token);
            token = strtok(NULL, " \t\n");
        }
    }

    fclose(fp);
    return 0;
}
\end{lstlisting}

\begin{tcolorbox}
    \mybox{运行结果}
    \begin{verbatim}
Talk
is
cheap
Show
me
the
code
Code
never
lies
comments
sometimes
do
Stay
Hungry
Stay
Foolish
	\end{verbatim}
\end{tcolorbox}

\newpage