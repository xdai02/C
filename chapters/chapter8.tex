\chapter{结构体}

\section{结构体}

\subsection{结构体(Structure)}

C中,数组是一种允许存储多个相同类型数据项的结构。结构体是另一种用户自定义的数据类型,它允许存储不同类型的数据项。\\

结构体的声明可以使用关键字struct,结构体名一般首字母大写。结构体的声明以【;】结束。结构体的声明通常定义为全局变量,这样就可以被多个函数所使用的了。

\vspace{-0.5cm}

\begin{lstlisting}[language=C]
struct struct_name {
    data_type var_name1;
    data_type var_name2;
    ...
};
\end{lstlisting}

通常会将用于描述同一个事物的变量定义成结构体。例如:

\begin{itemize}
	\item 日期(年、月、日)
	\item 坐标(横坐标、纵坐标)
	\item 学生信息(姓名、年龄、学号、成绩)
\end{itemize}

定义结构体变量时,不能只使用结构体名,需要加上struct关键字。\\

通过成员运算符【.】可以访问一个结构体之中的成员变量。\\

\mybox{结构体}

\begin{lstlisting}[language=C]
#include <stdio.h>

struct Date {
    int year;
    int month;
    int day;
};

int main() {
    struct Date date;
    date.year = 2021;
    date.month = 3;
    date.day = 12;

    printf("%d年%d月%d日\n", date.year, date.month, date.day);
    return 0;
}
\end{lstlisting}

\begin{tcolorbox}
	\mybox{运行结果}
	\begin{verbatim}
2021年3月12日
	\end{verbatim}
\end{tcolorbox}

\newpage

\section{typedef}

\subsection{typedef}

关键字typedef可以用来给数据类型定义别名,通过使用typedef可以简化结构体的声明,不用每次都加上struct关键字了。

\vspace{-0.5cm}

\begin{lstlisting}[language=C]
typedef struct [struct_name] {
    data_type var_name1;
    data_type var_name2;
    ...
} struct_name;
\end{lstlisting}

\vspace{0.5cm}

\mybox{typedef定义别名}

\begin{lstlisting}[language=C]
#include <stdio.h>

typedef struct Coordinate {
    double x;
    double y;
} Coordinate;

int main() {
    Coordinate coor;
    coor.x = 3.1;
    coor.y = 2.7;
    printf("(%.1f, %.1f)\n", coor.x, coor.y);
    return 0;
}
\end{lstlisting}

\begin{tcolorbox}
	\mybox{运行结果}
	\begin{verbatim}
(3.1, 2.7)
	\end{verbatim}
\end{tcolorbox}

\newpage

\section{结构体指针}

\subsection{结构体指针}

与数组不同,结构体变量的名字并不是结构体变量的地址,必须使用取地址运算符【\&】。\\

结构体也可以作为函数参数进行传递。如果是按值传递,那么在函数中会新创建一个结构体变量,并复制调用者的结构体的值。如果是按址传递,则需要传递结构体的指针。\\

C提供了一个间接引用运算符【->】,可以直接访问结构体指针所指的结构变量中的成员。\\

\mybox{倒数}

\begin{lstlisting}[language=C]
#include <stdio.h>
#include <stdlib.h>

// 分数
typedef struct Fraction {
    int numerator;       // 分子
    int denominator;     // 分母
} Fraction;

/**
    * @brief  倒数
    * @note   分母不能为0
    * @param  f: 分数结构体
    * @retval None
    */
void reciprocal(Fraction *f) {
    if(f->numerator == 0) {
        fprintf(stderr, "无法计算倒数\n");
    } else {
        int temp = f->numerator;
        f->numerator = f->denominator;
        f->denominator = temp;
    }
}

int main() {
    Fraction fraction = {2, 5};        // 2/5
    printf("%d/%d的倒数是", fraction.numerator, fraction.denominator);
    reciprocal(&fraction);
    printf("%d/%d\n", fraction.numerator, fraction.denominator);
    return 0;
}
\end{lstlisting}

\begin{tcolorbox}
	\mybox{运行结果}
	\begin{verbatim}
2/5的倒数是5/2
	\end{verbatim}
\end{tcolorbox}