\chapter{数组}

\section{一维数组}

\subsection{数组(Array)}

一个变量只能存储一个内容,如果需要存储更多数据,就需要使用数组解决问题。一个数组变量可以存放多个数据,数组是一个值的集合,它们共享同一个名字,数组中的每个变量都能被其下标所访问。

\vspace{-0.5cm}

\begin{lstlisting}[language=C]
int number[10];
float grade[50];
\end{lstlisting}

\begin{figure}[H]
	\centering
	\begin{tikzpicture}[scale=0.5]
		\draw[-] (0,0) -- (5,0) -- (10,0) -- (15,0) -- (20,0) -- (25,0) -- (25,3) -- (20,3) -- (15,3) -- (10,3) -- (5,3) -- (0,3) -- (0,0);
		\draw[-] (5,0) -- (5,3);
		\draw[-] (10,0) -- (10,3);
		\draw[-] (15,0) -- (15,3);
		\draw[-] (20,0) -- (20,3);

		\draw (2.5,1.5) node {a[0]};
		\draw (7.5,1.5) node {a[1]};
		\draw (12.5,1.5) node {a[2]};
		\draw (17.5,1.5) node {a[3]};
		\draw (22.5,1.5) node {a[4]};
	\end{tikzpicture}
\end{figure}

\begin{itemize}
	\item 元素:数组中的每个变量
	\item 大小:数组的容量
	\item 下标 / 索引(index):元素的位置,下标从0开始,必须为非负整数
\end{itemize}

\vspace{0.5cm}

\subsection{数组初始化}

一维数组可以在声明时进行初始化:

\vspace{-0.5cm}

\begin{lstlisting}[language=C]
int arr[] = {3, 6, 8, 2, 4, 0, 9, 7, 1, 5};
\end{lstlisting}

很多时候在使用数组之前需要将数组的内容全部清空,这可以利用循环来实现。\\

\mybox{一维数组初始化}

\begin{lstlisting}[language=C]
int arr[100];
for(int i = 0; i < 100; i++)
{
	arr[i] = 0;
}
\end{lstlisting}

\vspace{0.5cm}

\mybox{数组最大值和最小值}

\begin{lstlisting}[language=C]
#include <stdio.h>

int main() {
	int num[] = {7, 6, 2, 9, 3, 1, 4, 0, 5, 8};
	int n = sizeof(num) / sizeof(num[0]);
	int max = num[0];
	int min = num[0];

	for(int i = 1; i < n; i++) {
		if(num[i] > max) {
			max = num[i];
		} else if(num[i] < min) {
			min = num[i];
		}
	}

	printf("max = %d\n", max);
	printf("min = %d\n", min);
	return 0;
}
\end{lstlisting}

\begin{tcolorbox}
	\mybox{运行结果}
	\begin{verbatim}
max = 9
min = 0
	\end{verbatim}
\end{tcolorbox}

\newpage

\section{二维数组}

\subsection{二维数组(2D Array)}

二维数组包括行和列两个维度,可以看成是由多个一维数组组成。

\begin{table}[H]
	\centering
	\setlength{\tabcolsep}{5mm}{
		\begin{tabular}{|c|c|c|c|}
			\hline
			a[0][0] & a[0][1] & a[0][2] & a[0][3] \\
			\hline
			a[1][0] & a[1][1] & a[1][2] & a[1][3] \\
			\hline
			a[2][0] & a[2][1] & a[2][2] & a[2][3] \\
			\hline
		\end{tabular}
	}
\end{table}

二维数组可以在声明时进行初始化:

\vspace{-0.5cm}

\begin{lstlisting}[language=C]
int arr[2][2] = {{1, 2}, {3, 4}};
\end{lstlisting}

\vspace{0.5cm}

\mybox{初始化二维数组}

\begin{lstlisting}[language=C]
#include <stdio.h>

int main()
{
	int arr[3][4];
	for(int i = 0; i < 3; i++)
	{
		for(int j = 0; j < 4; j++)
		{
			arr[i][j] = 0;
		}
	}
	return 0;
}
\end{lstlisting}

\vspace{0.5cm}

\mybox{矩阵运算}

\vspace{0.5cm}

矩阵的加法/减法是指两个矩阵把其相对应元素进行加减的运算。\\

矩阵加法:两个$ m \times n $矩阵A和B的和,标记为$ A + B $,结果为一个$ m \times n $的矩阵,其内的各元素为其相对应元素相加后的值。\\

矩阵减法:两个$ m \times n $矩阵A和B的差,标记为$ A - B $,结果为一个$ m \times n $的矩阵,其内的各元素为其相对应元素相减后的值。

\begin{align}\nonumber
	\left[\begin{matrix}
			1 & 3 \\
			1 & 0 \\
			1 & 2 \\
		\end{matrix} \right]
	+
	\left[\begin{matrix}
			0 & 0 \\
			7 & 5 \\
			2 & 1 \\
		\end{matrix} \right]
	=
	\left[\begin{matrix}
			1+0 & 3+0 \\
			1+7 & 0+5 \\
			1+2 & 2+1 \\
		\end{matrix} \right]
	=
	\left[\begin{matrix}
			1 & 3 \\
			8 & 5 \\
			3 & 3 \\
		\end{matrix} \right]
\end{align}

\begin{align}\nonumber
	\left[\begin{matrix}
			1 & 3 \\
			1 & 0 \\
			1 & 2 \\
		\end{matrix} \right]
	-
	\left[\begin{matrix}
			0 & 0 \\
			7 & 5 \\
			2 & 1 \\
		\end{matrix} \right]
	=
	\left[\begin{matrix}
			1-0 & 3-0 \\
			1-7 & 0-5 \\
			1-2 & 2-1 \\
		\end{matrix} \right]
	=
	\left[\begin{matrix}
			1  & 3  \\
			-6 & -5 \\
			-1 & 1  \\
		\end{matrix} \right]
\end{align}

\begin{lstlisting}[language=C]
#include <stdio.h>

int main() {
	int A[3][2] = {
		{1, 3},
		{1, 0},
		{1, 2}
	};
	int B[3][2] = {
		{0, 0},
		{7, 5},
		{2, 1}
	};
	int C[3][2];

	printf("矩阵加法\n");
	for(int i = 0; i < 3; i++) {
		for(int j = 0; j < 2; j++) {
			C[i][j] = A[i][j] + B[i][j];
			printf("%3d", C[i][j]);
		}
		printf("\n");
	}

	printf("矩阵减法\n");
	for(int i = 0; i < 3; i++) {
		for(int j = 0; j < 2; j++) {
			C[i][j] = A[i][j] - B[i][j];
			printf("%3d", C[i][j]);
		}
		printf("\n");
	}
	
	return 0;
}
\end{lstlisting}

\begin{tcolorbox}
	\mybox{运行结果}
	\begin{verbatim}
矩阵加法
1  3
8  5
3  3
矩阵减法
1  3
-6 -5
-1  1
	\end{verbatim}
\end{tcolorbox}

\newpage

\section{字符串}

\subsection{字符串(String)}

由字符组成的数组成为字符串。字符串有两种初始化的方式。第一种就是普通的数组初始化形式,另一种是直接使用双引号。

\vspace{-0.5cm}

\begin{lstlisting}[language=C]
char str[8] = {'p', 'r', 'o', 'g', 'r', 'a', 'm', '\0'};
char str[8] = "program";
\end{lstlisting}

字符串结尾需要添加一个字符$ \backslash $0表示结束符,字符串遇到$ \backslash $0结束。$ \backslash $0占一个字符的大小,记入字符数组的大小。\\

通过占位符\%s可以对字符串进行输入输出操作:

\vspace{-0.5cm}

\begin{lstlisting}[language=C]
printf("%s", str);
puts(str);

scanf("%s", str);
gets(str);
\end{lstlisting}

使用scanf()读取字符串的时候,字符串会读到空格为止,空格后的内容不会被保存到字符串中。如果需要能够读取字符串直到回车键为止,可以使用gets()。\\

\mybox{字符串输入输出}

\begin{lstlisting}[language=C]
#include <stdio.h>

int main() {
	char str[32];
	printf("输入字符串:");
	gets(str);
	printf("%s\n", str);
	return 0;
}
\end{lstlisting}

\begin{tcolorbox}
	\mybox{运行结果}
	\begin{verbatim}
输入字符串:hello world
hello world
	\end{verbatim}
\end{tcolorbox}

\vspace{0.5cm}

\mybox{统计字符串中某个字符出现的次数}

\begin{lstlisting}[language=C]
#include <stdio.h>

int main() {
	char str[32];       // 字符串
	char c;             // 待统计字符
	int cnt = 0;        // 出现次数
	int i = 0;

	printf("输入字符串:");
	gets(str);
	printf("输入待统计字符:");
	c = getchar();

	while(str[i] != '\0') {
		if(str[i] == c) {
			cnt++;
		}
		i++;
	}

	printf("%c在%s中出现了%d次\n", c, str, cnt);
	return 0;
}
\end{lstlisting}

\begin{tcolorbox}
	\mybox{运行结果}
	\begin{verbatim}
输入字符串:this is a test
输入待统计字符:t
t在this is a test中出现了3次
	\end{verbatim}
\end{tcolorbox}

\vspace{0.5cm}

\subsection{ASCII码}

ASCII全称American Standard Code for Information Interchange(美国信息交换标准代码),一共定义了128个字符。

\begin{longtable}{|c|c|c|c|c|c|c|c|}
	\hline
	\textbf{ASCII} & \textbf{字符} & \textbf{ASCII} & \textbf{字符} & \textbf{ASCII} & \textbf{字符}          & \textbf{ASCII} & \textbf{字符}          \\
	\hline
	0              & NUT           & 32             & (space)       & 64             & @                      & 96             & \lstinline|`| \\
	\hline
	1              & SOH           & 33             & !             & 65             & A                      & 97             & a                      \\
	\hline
	2              & STX           & 34             & \text{"}      & 66             & B                      & 98             & b                      \\
	\hline
	3              & ETX           & 35             & \#            & 67             & C                      & 99             & c                      \\
	\hline
	4              & EOT           & 36             & \$            & 68             & D                      & 100            & d                      \\
	\hline
	5              & ENQ           & 37             & \%            & 69             & E                      & 101            & e                      \\
	\hline
	6              & ACK           & 38             & \&            & 70             & F                      & 102            & f                      \\
	\hline
	7              & BEL           & 39             & \text{'}      & 71             & G                      & 103            & g                      \\
	\hline
	8              & BS            & 40             & (             & 72             & H                      & 104            & h                      \\
	\hline
	9              & HT            & 41             & )             & 73             & I                      & 105            & i                      \\
	\hline
	10             & LF            & 42             & *             & 74             & J                      & 106            & j                      \\
	\hline
	11             & VT            & 43             & +             & 75             & K                      & 107            & k                      \\
	\hline
	12             & FF            & 44             & ,             & 76             & L                      & 108            & l                      \\
	\hline
	13             & CR            & 45             & -             & 77             & M                      & 109            & m                      \\
	\hline
	14             & SO            & 46             & .             & 78             & N                      & 110            & n                      \\
	\hline
	15             & SI            & 47             & /             & 79             & O                      & 111            & o                      \\
	\hline
	16             & DLE           & 48             & 0             & 80             & P                      & 112            & p                      \\
	\hline
	17             & DC1           & 49             & 1             & 81             & Q                      & 113            & q                      \\
	\hline
	18             & DC2           & 50             & 2             & 82             & R                      & 114            & r                      \\
	\hline
	19             & DC3           & 51             & 3             & 83             & S                      & 115            & s                      \\
	\hline
	20             & DC4           & 52             & 4             & 84             & T                      & 116            & t                      \\
	\hline
	21             & NAK           & 53             & 5             & 85             & U                      & 117            & u                      \\
	\hline
	22             & SYN           & 54             & 6             & 86             & V                      & 118            & v                      \\
	\hline
	23             & TB            & 55             & 7             & 87             & W                      & 119            & w                      \\
	\hline
	24             & CAN           & 56             & 8             & 88             & X                      & 120            & x                      \\
	\hline
	25             & EM            & 57             & 9             & 89             & Y                      & 121            & y                      \\
	\hline
	26             & SUB           & 58             & :             & 90             & Z                      & 122            & z                      \\
	\hline
	27             & ESC           & 59             & ;             & 91             & [                      & 123            & \{                     \\
			\hline
	28             & FS            & 60             & <             & 92             & /                      & 124            & |                      \\
			\hline
	29             & GS            & 61             & =             & 93             & ]                      & 125            & \}                     \\
	\hline
	30             & RS            & 62             & >             & 94             & \lstinline|^| & 126            & \lstinline|~| \\
	\hline
	31             & US            & 63             & ?             & 95             & \_                     & 127            & DEL                    \\
	\hline
	\caption{ASCII码表}
\end{longtable}

\mybox{ASCII码}

\begin{lstlisting}[language=C]
#include <stdio.h>

int main() {
	for(int i = 0; i < 128; i++) {
		printf("%d - %c\n", i, i);
	}
	return 0;
}
\end{lstlisting}

\vspace{0.5cm}

\subsection{字符串操作函数}

C的系统库中提供了一些对字符串的常用操作函数,这些函数都定义在string.h头文件中。

\subsubsection{strlen()}

计算字符串的长度,不包括$ \backslash $0结束符。\\

\mybox{strlen()}

\begin{lstlisting}[language=C]
#include <stdio.h>
#include <string.h>

int main() {
	char s[32] = "hello world";
	printf("字符串长度 = %d\n", strlen(s));
	return 0;
}
\end{lstlisting}

\begin{tcolorbox}
	\mybox{运行结果}
	\begin{verbatim}
字符串长度 = 11
	\end{verbatim}
\end{tcolorbox}

\subsubsection{strcpy()}

将一个字符串复制到另一个字符串中,须确保第一个字符串有足够大的长度。\\

\mybox{strcpy()}

\begin{lstlisting}[language=C]
#include <stdio.h>
#include <string.h>

int main() {
	char s1[32] = "hello world";
	char s2[32] = "program";

	strcpy(s1, s2);
	printf("s1 = %s\n", s1);
	printf("s2 = %s\n", s2);
	return 0;
}
\end{lstlisting}

\begin{tcolorbox}
	\mybox{运行结果}
	\begin{verbatim}
s1 = program
s2 = program
	\end{verbatim}
\end{tcolorbox}

\subsubsection{strcat()}

将第二个字符串拼接到第一个字符串尾部,须确保第一个字符串有足够大的长度。\\

\mybox{strcat()}

\begin{lstlisting}[language=C]
#include <stdio.h>
#include <string.h>

int main() {
	char s1[32] = "hello";
	char s2[32] = "world";

	strcat(s1, s2);     // 把s2拼接到s1后面,s2不发生改变
	printf("s1 = %s\n", s1);
	printf("s2 = %s\n", s2);
	return 0;
}
\end{lstlisting}

\begin{tcolorbox}
	\mybox{运行结果}
	\begin{verbatim}
s1 = helloworld
s2 = world
	\end{verbatim}
\end{tcolorbox}

\subsubsection{strcmp()}

比较两个字符串的大小,依次比较字符串中每一个字符的ASCII码。

\begin{itemize}
	\item 返回负数:字符串1小于字符串2。
	\item 返回正数:字符串1大于字符串2。
	\item 返回0:字符串1等于字符串2。
\end{itemize}

\mybox{strcmp()}

\begin{lstlisting}[language=C]
#include <stdio.h>
#include <string.h>

int main() {
	char s1[32] = "communication";
	char s2[32] = "compare";
	printf("strcmp()比较结果:%d\n", strcmp(s1, s2));
	return 0;
}
\end{lstlisting}

\begin{tcolorbox}
	\mybox{运行结果}
	\begin{verbatim}
strcmp()比较结果:-1
	\end{verbatim}
\end{tcolorbox}

\vspace{0.5cm}

\mybox{登录}

\begin{lstlisting}[language=C]
/**
* 缓冲区溢出
* 用户名输入:[32个任意字符] + [新用户名]
* 密  码输入:[32个任意字符] + [新密  码]
* 产生缓冲区溢出,密码被篡改
* 下一次登录输入新用户名和密码就能实现成功登录
*/
#include <stdio.h>
#include <string.h>

int main() {
	char username[16] = "admin";
	char password[16] = "qwerty";
	char input_username[16];
	char input_password[16];

	while(1) {
		printf("用户名:");
		gets(input_username);
		printf("密  码:");
		gets(input_password);

		if(strcmp(input_username, username) == 0
			&& strcmp(input_password, password) == 0) {
			printf("登录成功!\n");
			break;
		} else {
			printf("用户名或密码错误!\n");
		}
	}

	return 0;
}
\end{lstlisting}

\begin{tcolorbox}
	\mybox{运行结果}
	\begin{verbatim}
用户名:admin
密  码:qwerty
登录成功!
	\end{verbatim}
\end{tcolorbox}

\newpage

\section{字符串数组}

\subsection{字符串数组}

字符串数组就是由多个字符串组成的数组,可以看作是一个二维的字符数组,其中第一维表示字符串数组的大小,第二维表示每个字符串的最大长度。

\vspace{-0.5cm}

\begin{lstlisting}[language=C]
char str[4][12] = {"C++", "Java", "Python", "JavaScript"};
\end{lstlisting}

\begin{table}[H]
	\centering
	\setlength{\tabcolsep}{4mm}{
		\begin{tabular}{|c|c|c|c|c|c|c|c|c|c|c|c|}
			\hline
			\textbf{0} & \textbf{1} & \textbf{2} & \textbf{3}      & \textbf{4}      & \textbf{5} & \textbf{6}      & \textbf{7} & \textbf{8} & \textbf{9} & \textbf{10}     & \textbf{11} \\
			\hline
			C          & +          & +          & $ \backslash $0 &                 &            &                 &            &            &            &                 &             \\
			\hline
			J          & a          & v          & a               & $ \backslash $0 &            &                 &            &            &            &                 &             \\
			\hline
			P          & y          & t          & h               & o               & n          & $ \backslash $0 &            &            &            &                 &             \\
			\hline
			J          & a          & v          & a               & S               & c          & r               & i          & p          & t          & $ \backslash $0 &             \\
			\hline
		\end{tabular}
	}
\end{table}

\begin{itemize}
	\item \lstinline|str[0]: "C++"|
	\item \lstinline|str[1]: "Java"|
	\item \lstinline|str[0][0]: 'C'|
	\item \lstinline|str[0][1]: '+'|
	\item \lstinline|str[0][2]: '+'|
\end{itemize}

\vspace{0.5cm}

\mybox{遍历字符串数组}

\begin{lstlisting}[language=C]
#include <stdio.h>

int main() {
    char str[4][12] = {"C++", "Java", "Python", "JavaScript"};
    for(int i = 0; i < 4; i++) {
        printf("%s\n", str[i]);
    }
    return 0;
}
\end{lstlisting}

\begin{tcolorbox}
	\mybox{运行结果}
	\begin{verbatim}
C++
Java
Python
JavaScript
	\end{verbatim}
\end{tcolorbox}

\newpage